\chapter{到達目標}

\section{本プロジェクトのおける目標}
  陣川あさひ町会は1.2の通り, 様々な問題を抱えている.このことから, 本プロジェクトでは2つに分けて目標を設定した.

\subsection{目標1}
  本プロジェクトは、陣川あさひ町会で開催されるイベントをシステム内で一覧で表示し, 表示したイベントに対して参加申し込みを可能にすることで町会側がシステム内で参加者の情報も管理できるようにすることを1つ目の目標として設定した.前述の通り陣川あさひ町会は開催予定のイベントを一覧で見ることができないという問題や, Facebook やLINE@ といった現在利用してるSNS から参加申し込みすることができないという問題を抱えてる.そのためイベントの一覧表示と表示されたイベントへの参加申込みが可能になることで町民のSNS 間の移動という手間が省け, 町会側は参加者の情報が紙面ではなく, システムにまとめられた状態になるため従来より参加者の管理が楽になり, 前述の問題を解決することが可能になると考えたためである.\\

\subsection{目標2}
  2つ目の目標は, 老若男女問わず, 全ての年齢の方にとって使いやすいアプリにすることである.前述の通り, 陣川あさひ町会は1200世帯中1000世帯が加盟しており、年齢の幅も大きいため, ある一定の年齢層だけが利用しやすいアプリを作成してしまうと, 他の年齢層が使いにくくなってしまう.このことから, 町内会のアプリがあるのに使われなくなる問題がある.そこでこの問題を未然に防ぐため, 町内会と定期的に会議を行いレビューと改善のサイクルを繰り返すことで, 誰にとっても使いやすいアプリにすることを目標とした.\\

\bunseki{伊藤泰斗}