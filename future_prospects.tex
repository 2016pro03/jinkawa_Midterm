\chapter{今後の展望}

\section{動作の不安定を解消}
11月18日の町会打ち合わせにて役員に開発途中のじぷりを利用してもらったところ,iOS, Android共に動作が不安定であることがわかった.
iOSでは,イベントの新規作成をすることができるが,イベントの更新をすることができなかった.Androidでは,イベントの新規作成時に写真の追加をすることができなかった.
それぞれの原因を特定して解消していく.解消後は1月11日~20日に町会に向けてじぷりを実際に利用してもらい,そこで出た要望や問題を解決し最終調整を行い,1月31日までにリリースする予定である.
\bunseki{船木綾香}

\section{無償サーバサービスの利用}
現在無償サーバサービスのmBaasを利用している.町会の意向により今後も無償のサービスを利用していく.
しかし,じぷりのサーバへのリクエスト回数が無料の範囲を超えてしまう場合は,無償のサーバでは運用することは難しいため町会で独自のサーバを立ち上げる可能性がある.

\section{今後の機能の追加と改善}
私たちは陣川町民の多くの人にじぷりを長く利用してもらいたいと考えている.
そのため,今後は今期に実装できなかった機能を追加していく予定である.
具体的な機能は,イベントへの参加申し込みが来たらイベント管理者へ通知がされる機能,特定のイベントの参加者のみにメッセージを送る機能,
イベントへの参加者自身で参加の申し込みを取り消す機能,イベントへの参加を取り消した人をイベント管理者へ通知がされる機能,イベント開催する担当部署ごとに通知のオンオフができる機能,
イベントが新しく追加されたらじぷりを持っている全ユーザへ通知がされる機能,自分が参加する予定のイベントを確認できる機能が挙げられる。
また,現在のじぷりはスマートフォンの操作に慣れていない人でも利用しやすいUIを目指して作っているが,今後は幅広い年齢層の方や町民外の方にでも利用してもらえるようなUIに改善していく予定である。
\bunseki{船木綾香}
