\chapter{今後の予定と展望}

\section{予定}
\subsection{プロトタイプの作成}
「じぷり」のプロトタイプでは、役員によるイベント管理機能、お知らせ機能、通知機能を実装予定である。イベント管理機能は、町会のイベントの作成、編集、削除ができる機能である。お知らせ機能では、イベントの情報や、ゴミ収集情報や天気予報などの生活情報が発信できる機能である。通知機能では、アプリケーションの使用者全員にイベントやお知らせの情報が通知される機能である。
\subsection{プロトタイプの評価}
8月6日に町会で開催される納涼まつりに参加し、実際に町民を対象に前述した、「じぷり」のプロトタイプのデモを行う。デモ終了後、町民にアプリケーションに関するアンケートに回答してもらう予定である。アンケートは、町民の情報機器利用に関する意見や作成したプロトタイプの意見を収集するために行う。このアンケートの結果を今後のアプリケーション開発に反映させていく。
\bunseki{森島帆南}

\section{今後の展望}
\subsection{既存の情報発信手段との連携について}
既存の町会のイベント情報発信手段として利用されている、FacebookとLINE@との連携機能を追加していきたい。例として、「じぷり」にイベントが作成されたら、その情報がFacebookとLINE@に投稿されるという機能が考えられる。
\subsection{町民にとっての魅力的なコンテンツの考案}
6月23日の第2回提案にて、町会から町民がアプリケーションを利用したくなるようなコンテンツをアプリケーションに追加して欲しいという要望があった。これに対して、過去のイベントの写真が確認できるWebページとアプリケーションをリンクさせる方法を提案したい。
\bunseki{森島帆南}
