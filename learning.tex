\chapter{学び}

\section{メンバー間の情報共有}
メンバーで議論をしているときに、メンバー全員が個人のパソコンに向かって話している状態になっていた。そのため、メンバー全員が同じ認識を持っているかを確認しないまま話を進めてしまい、個人個人の認識に差異が生じてしまった。改善策としてスクリーンを用いての画面共有と席替えを行った。スクリーンを用いて、議論の進行に必要な画面を投影することで、全員が同じ認識で議論を進めることができた。また、席替えではリーダーが真ん中に座りメンバー全員に質問を振りながら議論を進めることで、メンバー間の認識に差異が生まれにくくした。
\section{計画的なスケジューリング}
メンバー個人個人の忙しさを考慮せずにタスクを割り振ってしまったことから、全体の進捗を遅らせてしまった。また、タスク1つ1つに対してしっかりと期日を設けなかったので、中間発表の2週間前にタスクが処理しきれないほど増えてしまった。これらの反省から、後期の活動では全員のスケジュールをメンバー間で確認し、適切なタスクの量を割り振ることにする。また、Redmineの利用を定着させることで、タスクの進捗状況を全員が把握できるようにする。
\bunseki{森島帆南}
