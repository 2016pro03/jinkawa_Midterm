\chapter{開発プロセス}

\section{ヒアリング}%例:レビュー内容
 我々は陣川あさひ町会の置かれている現状を明らかにするため、5月12日に町会に対してヒアリングを行った。
その結果、主に3つの課題が明らかになった。
1つ目の課題は、イベント情報を町民と役員で共有したいのだが、その中でも役員だけで共有したい情報を町民に知られずに共有することが出来ないことである。
なぜなら、役員だけで共有したい情報は町民にとって不要だからである。
2つ目の課題は、イベントへの参加申し込みがオンライン上で出来ないことである。
なぜなら、Facebookでイベントページを立ち上げ参加者を募ると、参加者から他の参加者の情報が閲覧できてしまい、個人情報が漏れてしまうからである。
3つ目の課題は早期にイベント開催情報を発信することが出来ないことである。
そこで我々は上記3つの問題を解決するアプリケーションを考えることとした。

\section{アプリアイデアの考案}
まず、1つ目の課題を解決するためには、町会役員間で役員会議等の情報を共有することが出来るような機能が必要である。
2つ目の問題を解決するためには個人情報を役員しか知ることが出来ないような機能が必要である。
3つ目の課題を解決するためには、イベントの開催が決まり次第詳細な情報がなくてもイベントを投稿することが出来る機能が必要である。
したがって、我々は以上3つの機能を備えたアプリケーションイメージを考案した。

\section{第1回提案・レビュー}
 5月30日に我々が考えたアプリケーションの画面イメージを町会に提案した。
その結果、iOS、Android、Webアプリ3つに対応可能なアプリケーション開発を行うことが決定した。
また、我々の考案したアプリケーションイメージについて、3つのレビューを頂いた。
1つ目は、イベントへの参加申し込み情報入力フォームについて、参加者の情報として「名前」「性別」「年齢」「住所」「電話番号」が必要であるとのレビューである。
なぜなら、参加者の名簿を市役所に提出する必要があり、その名簿には上記5つの情報が必要だからである。
2つ目は、アプリケーションをインストールしてくれた人が、すぐイベントを確認できるように起動時の画面はログイン画面にしないで欲しいとのレビューである。
3つ目は、イベント作成情報入力フォームでには「定員」の項目を追加して欲しいとのレビューである。

\section{アプリアイデアの改善}
 頂いたレビューの内容を利用してアプリケーションを改善した。
その改善内容は、イベント参加申し込みフォームに「名前」「性別」「年齢」「住所」「電話番号」を入力出来るような画面にし、
起動時にはログイン画面を用意せず、イベント作成フォームには「定員」の項目を追加した。
その改善案に対して、教員より役員と町民でイベントカレンダーを共有することで本当に問題を解決できるのかと指摘を受け、
アプリケーションについて再考し改善を図った。
その結果、カレンダーを用いて開催予定のイベントを表示するのではなく、開催予定のイベントを直近のものから順にリスト表示することにした。
なぜなら、カレンダー表示では来月の予定などがひと目で確認することが出来ないからである。

\section{第2回提案・レビュー}
 6月23日に我々は町会に対して改善したアプリケーションイメージを提案した。
その結果、画面ごとに詳細なレビューを頂いた。
例として、イベント一覧でイベントをタップすると画面いっぱいにイベントの詳細情報が表示されるようにして欲しいという要望、
イベント作成の画面にアプリの所有者全員に通知するかしないかのチェックボックスを設けて欲しいという要望が挙げられる。
また、町民が利用したくなるようなコンテンツを追加して欲しいという要望を得た。
例として、過去のイベントの写真が確認できるWebページとアプリケーションとリンクさせることが挙げられる。
\bunseki{永井陽太}
