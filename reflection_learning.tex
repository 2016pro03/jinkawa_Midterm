\chapter{前期の振り返りと学び}

\section{前期の振り返り}
我々は7月13日にこれまでの活動の振り返りを行った。
はじめに、5月から我々が行ってきたこと、その際に感じたこと、
心に残ったアドバイスについてそれぞれ黄色、緑、赤の付箋に書き出した。
その後、それらを2枚の模造紙に期間ごとに貼り付けてグループメンバ全員で見返した。
その次に、我々は今までの活動の中で良い点、悪い点、これからやっていきたいことを話し合った。
良い点として、メンバ間で積極的にコミュニケーションを取り合うことでメンバの関係性を良好に保てたことが挙げられた。
悪い点として、メンバの予定を考慮することなくスケジュールの決定を行った点、各作業に要する時間の想定が困難だったため、
メンバに負担がかなり掛かってしまった点が挙げられた。
これからやっていきたいこととしては、TAや教員等相談できる人がいるという環境を有効に活用することで、活動の行き詰まる時間を削減することが挙げられた。
我々はこの振り返りを通して5月からの活動を客観的に見ることができた。
\bunseki{森島帆南}

\section{学び}
\subsection{メンバ間の情報共有}
メンバで議論をしているときに、メンバ全員が個人のパソコンに向かって話している状態になっていた。そのため、メンバ全員が同じ認識を持っているかを確認しないまま話を進めてしまい、個人個人の認識に差異が生じてしまった。改善策としてスクリーンを用いての画面共有と席替えを行った。スクリーンを用いて、議論の進行に必要な画面を投影することで、全員が同じ認識で議論を進めることができた。また、席替えではリーダーが真ん中に座りメンバ全員に質問を振りながら議論を進めることで、メンバ間の認識に差異が生まれにくくした。
\subsection{計画的なスケジューリング}
メンバ個人個人の忙しさを考慮せずにタスクを割り振ってしまったことから、全体の進捗を遅らせてしまった。また、タスク1つ1つに対してしっかりと期日を設けなかったので、中間発表の2週間前にタスクが処理しきれないほど増えてしまった。これらの反省から、後期の活動では全員のスケジュールをメンバ間で確認し、適切なタスクの量を割り振ることにする。また、Redmineの利用を定着させることで、タスクの進捗状況を全員が把握できるようにする。
\bunseki{森島帆南}
