\chapter{振り返り}

\section{前期の振り返り}
7月13日にプロジェクト学習前期活動の振り返りを行った. はじめに, 5月から我々が行ってきたこと, その際に感じたこと, 心に残ったアドバイスについてそれぞれ黄色,緑,赤の付箋に書き出した.その後,
それらを2枚の模造紙に期間ごとに貼り付けてグループメンバ全員で見返した.その次に,我々は今までの活動の中で良い点,悪い点,これからやっていきたいことを話し合った.
良い点として,メンバ間で積極的にコミュニケーションを取り合うことでメンバの関係性を良好に保てたことが挙げられた.悪い点として,メンバの予定を考慮することなくスケジュールの決定を行った点,
各作業に要する時間の想定が困難だったため,メンバに負担がかなり掛かってしまった点が挙げられた.これからやっていきたいこととしては, TAや教員等相談できる人がいるという環境を有効に活用することで,
活動の行き詰まる時間を削減することが挙げられた.
\bunseki{森島帆南}

\section{後期の振り返り}
12月21日に前期と後期を含めたプロジェクト活動全体で起きたことを深く掘り下げるために振り返りを行った. はじめに, 前期のKPTを小さい模造紙にまとめた.
次に, 後期のマイルストーンごとに, チームや個人で取り組んだこと, その時に思っていたこと, その要因をそれぞれ緑, 黄色, 赤の付箋に書き出し, 別の模造紙に貼り付けた.
その後, メンバで貼り付けた情報を共有し, これまでの活動の良かった点, 悪かった点について話し合った. 良い点として,以下の2点が挙げられた. 
\begin{enumerate}
    \item Trelloを用いて, スプリントバックログとプロダクトバックログの作成・管理を行ったこと
    \item 町会打ち合わせで議事録ドリブンを導入したこと
\end{enumerate}
Trelloを使い始めてから, 残りのタスクの数が可視化されて全体の進捗が把握しやすくなり, タスクの管理の効率が良くなった. 議事録ドリブンを導入した理由は, 町会打ち合わせで要望がたくさん出てしまい,
まとめるのが大変であったためである. 議事録ドリブンを導入したことで, その場で要望を整理し, 役員の方たちと共有しながら確認することができ, 普段より要望の整理にかかる時間を短縮することができた.
以上の2点がメンバ全員が後期の活動の中で特に良かったと感じることであった. また, メンバの1人からは, 自身で開発しているため, 操作性についてまったく気にしていなかったが,
実際に先方に使ってもらった時に操作に手間がかかっていた部分があることが発覚し, 改めてUIの大切さを身をもって感じたという意見があった.
悪い点として, 計画的なスケジュールを組み立てられなかったことが1番の反省点に挙げられた. スケジュールを立てる際に, 自分たちの実力が不明だったのでタスクの期日を定めることができなかった.
しかし, 最終的な目標を立ててしっかりとした期日を設けて, それまでにタスクを終わらせる方針をとっていれば, スケジュールが詰まらずに余裕をもって活動を行えたと考えられる.
\bunseki{森島帆南}

\chapter{学び}
\section{先方との打ち合わせ}
先方と行ってきた打ち合わせに関しての3つの問題があった.
1つ目は十分に打ち合わせの事前準備が行えていなかったことである. 事前に打ち合わせの目的を明確にして, 先方に必ず確認すべきことを定めておくことの重要性を学んだ.
また, 先方に見せる資料のピアレビューを必ず全員で行うことも大切であると学んだ. 実際に, メンバの1人が作成した資料を確認しておらず, 打ち合わせ直前に確認したところ, 説明不十分の部分があり,
資料訂正に時間を割かれてしまい, 打ち合わせの練習ができないことがあった. 事前に全員が資料に目を通していれば, 余裕を持ってより先方にとって理解しやすい資料を作成できる可能性があった.

2つ目は, 打ち合わせの進め方の効率が悪いことである. 打ち合わせ中に先方から要望が数多く出てしまいまとめるのが大変であった. この問題から, 打ち合わせを進行しながら議事録を書いていく議事録ドリブンを導入した.
議事録ドリブンを導入したことで, 要望を整理することができる. また, 打ち合わせの最後に先方の方を含め全員で打ち合わせの内容を振り返ることが簡単にできるようになり, 打ち合わせの時間も短縮することができた.

3つ目は, 先方の要望の分析が不足していたことである.
前期では, 先方からの数多くの要望をメンバ全員ではなく, 各々で重要な部分をまとめようとしてしまった. 先方の要望の本質の分析不足で先方にとって本当に使いやすいとは言えないものを提案してしまった.
教員からも先方にとって満足のできるものになっているのか見直した方が良いとのレビューを受けた. 我々はもう1度先方の要望を洗い出し, 先方の本当に求めているものを別のアプローチから解決できないか分析し直した.
また, 全ての機能を書き出した後にそれぞれの優先順位をつけることで, 機能が可視化されて同時に進行した方が作業の効率が良くなる機能なども確認することができた.
\bunseki{森島帆南}
\section{メンバ間の情報共有}
チーム全体でのメンバ間の情報共有に関して, 2つの問題が起きてしまった.
1つ目は, チーム全員での議論時に個人個人の認識に差異が生じてしまうことがあった. メンバ全員が個人のパソコンに向かって話している状態になっていることが多くあり,
同じ認識を持っているかを確認しないまま話を進めてしまったからである. 改善策としてスクリーンを用いての画面共有と席替えを行った. スクリーンを用いて,議論の進行に必要な画面を投影することで,
全員が同じ認識で議論を進めることができた.また,席替えではリーダーが真ん中に座りメンバ全員に質問を振りながら議論を進めることで,メンバ間の認識に差異が生まれにくくすることができた.

2つ目は, チーム全体のタスクの進捗が把握できていなかったことであった. 前期では, メンバ個人個人の忙しさと期日を考慮せずにタスクを割り振ってしまい進捗に偏りが出てしまった.
この反省点から, チーム全体のタスクの進捗が把握できていなかったことであったとわかった. 後期から, スプリントバッグログとプロダクトバッグログをTrelloで管理し始め,
プロジェクト全体のタスクが可視化されてお互いの役割を把握しやすくし進捗が遅れているメンバのサポートをしやすくなった.
\bunseki{森島帆南}

\section{システムコースとデザインコースの考え方の違い}
本チームは情報システムコース1名と情報デザインコース2名と高度ICTコース2名で構成されていたため, 2つの問題が起きてしまった.
1つ目は, チーム全体のコミュニケーションが不足してしまったことであった. システム班とデザイン班に分かれて活動するようになり, チーム全体でのコミュニケーションを取る時間が減ってしまった.
結果, わからないことを確認する機会がなくわからないことや疑問におもっていることをそのままにしてしまっていた. そこで我々は, プロジェクト活動の最初に前回までのタスクの進捗報告をすることを始め,
全体でのコミュニケーションをとりやすくすることでタスクでのわからないことを共有し解決することができた.

2つ目は, システム班とデザイン班の意思疎通がうまく噛み合わないことであった.
デザイン班がじぷりの仕様が大きく変わる案を考え, システム班に変更内容を伝え承諾を得た. しかし, しばらく経った後に仕様の変更内容がシステム班に正しく伝わっていないことがわかり,
説明し直さなければいけないことがあった. お互い得意とする分野が違い専門的な知識や考え方が大きく異なるためデザインの感覚の言葉選びではなく,
システム班の人たちが理解しやすいように仕様変更内容の理由を伝えるべきであった. こうすることで, 専門的な知識が異なってもお互いの意思疎通ができることを学んだ.
\bunseki{森島帆南}

\section{スケジュール管理}
前期, 後期を通してスケジューリングの粗さが目立った. 自分たちの実力や, 忙しさを考慮せずにスケジュールを立てた結果リスケジューリングを余儀なくされる場面が多く見られた. さらに, リスケジューリングする際の各タスクの工期の見積もりが甘く, リスケジューリングを繰り返す場面も多く見られた. また, リスケジューリングした後に町会との打ち合わせが企画されたり, 仕様の変更などが要求された結果スケジュールが大きく乱れた場面も見られた.
これらの経験からスケジュールを立てる時は, 各タスクに対して余裕を持って工期をを見積もることと, リスケジューリングをする際は今あるタスクへの影響度などを考慮した上で, リスケジューリングすべきだということを学んだ. また, 作業に必要な時間, 担当者, 作業の洗い出し, 期限を設定しいつでも確認できるようにすることで, 忘れることなく作業に取り組むことができた.
\bunseki{森島帆南}
