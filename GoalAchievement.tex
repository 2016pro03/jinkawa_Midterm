\chapter{到達目標に対する評価}

\subsection{目標1についての評価}
目標1は、陣川あさひ町会で開催されるイベントをシステム内で一覧で表示し, 表示したイベントに対して参加申し込みを可能にすることで町会側がシステム内で参加者の情報も管理できるようにすることとしていた.このイベント一覧表示や参加申し込み機能、参加者管理機能について, 発表会で得られたレビューより, イベントを見てから参加するまでの過程がスムーズで便利だと思ったとの評価を得た.以下に, 最終発表会で得られたアンケートの「よりアプリを使いやすくするにはどのようにしたらいいか?」の項目から、上記で述べた機能についてのコメントを抜粋した.

\begin{itemize}
    \item ログイン機能があるともっと便利になると思った.
    \item より家族間で扱いやすいようにするといいと思った.
\end{itemize}

これらのコメントから, 目標1についてはおおむね到達目標に到達しているものと評価できる.アンケートではこれらの機能をより良くしていくためのアドバイスが多かった. そのため今後は最終成果発表会で得たレビューを参考により町民や町会にとって使いやすい機能の開発を進めていく.

\subsection{目標2についての評価}
目標2は, 老若男女問わず, 全ての年齢の方にとって使いやすいアプリにすることとしていた.この目標にについて, 発表会で得られたレビューより, 先方とたくさん打ち合わせをしている様子がアプリから伝わり, 洗練されていると感じたとの評価を得た.以下に最終発表会で得られたアンケートの「発表内容」の項目から、上記で述べた機能についてのコメントを抜粋した.

\begin{itemize}
    \item 見た目がシンプルで迷わず操作できると思う.
    \item ヒアリングの効果が出ていると思う
\end{itemize}

これらのコメントのみ考慮すると目標2については到達目標に達成できていると評価できる.しかし「町民や使う画面か管理者が使う画面かわからない」や「予定イベントが見づらい」など課題の指摘も受けた.そのため, 両方のコメントを考慮すると目標2については到達目標に達成したと言えない部分が多く残ったと言える.
今後はアンケートで評価を頂いた項目を参考に町会や町民にとって使いやすい機能やデザインを考慮してリリースに向けて開発を進めていく.