\chapter{中間発表}
​
\section{発表形式}
7月8日に行われた中間発表では, 各グループが行ってきた活動を詳細に伝え, 後期の活動に活かせるレビューをもらうことを目的とした.
そのため全体ポスター2分, 各グループのポスターとデモを含めた発表を12分間並行して発表を行った.
\bunseki{伊藤泰斗}

\section{レビュー内容}%例:レビュー内容
%必要なら下のsubsectionを用いて小見出しをつかう
\subsection{発表方法についての評価と反省}%:発表技法について
以下に, 中間発表会で行ったアンケートの「発表技術について」の項目から, メンバ間で精査した結果, 最終成果発表にも取り入れたいコメントを抜粋した.
\begin{itemize}
  \item デモがプロトタイプであることを伝えないと, 実装したものだと勘違いしてしまう.
  \item もう少しスラスラ話せていたら分かりやすかったと感じた.
\end{itemize}
    上記より, 伝える情報とポスターセッションの練習の不足が伺える.
    しかし, 「とても喋りに安定感があるなと感じた」との評価も受けた. 最終成果発表の際にはすべて開発したアプリケーションでデモを行い,
    ポスターセッションをする人全員がスラスラと話せるくらいに練習を行っていく.

\subsection{発表内容についての評価と反省}
    「発表内容について」の項目から後期の開発や発表において考慮すべきコメントを抜粋した.
\begin{itemize}
  \item 陣川町民に使ってもらうためのプロモーションの方法を考えたほうが良い
  \item クーポンなど, ユーザを得る工夫が欲しい
  \item ユーザにより沿って開発していく中で生起した出来事を大切に記述して欲しい
\end{itemize}
    上記より, 2つの見落としが伺えた. 1つ目はユーザに使ってもらうための考慮をしていなかったことである.
    メンバ全員が使ってもらえることを前提として考えていることである. しかし実際には使ってもらえることは前提ではないため,
    どのようにして使ってもらうのかを考える必要がある. 2つ目は, 本アプリケーションにユーザにとって魅力的な優位性が必要であることである.
    認知されていてもユーザにとって使いたいものでなければ使ってもらうことができない. そのため, 最終成果発表までにプロモーションの方法を考え,
    使ってもらうための工夫を本アプリケーションに追加することでユーザを獲得していきたい.
\bunseki{伊藤泰斗}
