\chapter{中間発表}
​
\section{発表形式}
7月8日に行われた中間発表では、各グループが行ってきた活動を詳細に伝え、後期の活動に活かせるレビューをもらうことを目的とした。そのため全体ポスター2分、各グループのポスターとデモを含めた発表を12分間並行して発表を行った。
\section{レビュー内容}%例:レビュー内容
%必要なら下のsubsectionを用いて小見出しをつかう
\subsection{発表方法についての評価と反省}%:発表技法について
以下に、中間発表会で行ったアンケートの「発表技術について」の項目から、メンバー間で精査した結果、最終成果発表にむ取り入れたいコメントを抜粋した。

    ・デモがプロトタイプであることを伝えないと、実装したものだと勘違いしてしまう。
    ・もう少しスラスラ話せていたら分かりやすかったと感じた。

    これらのコメントから、伝える情報の不足とポスターセッションの練習不足が伺える。しかしながら、「とても喋りに安定感があるなと感じた」との評価も受けた。最終成果発表の際には全て開発したアプリケーションでデモを行い、ポスターセッションをする人全員がスラスラと話せるくらいに練習を行っていく。

\subsection{発表内容についての評価と反省}
    「発表内容について」の項目から後期の開発や発表において欠かせないとグループ内で精査したコメントを抜粋した。

    ・陣川町民に使ってもらうためのプロモーションの方法を考えたほうが良い
    ・クーポンなど、ユーザを得る工夫が欲しい
    ・ユーザにより沿って開発していく中で生起した出来事を大切に記述して欲しい

    これらのコメントから、2つの見落としが伺えた。1つ目はユーザに使ってもらうための考慮をしていなかったことである。メンバー全員が使ってもらえることを前提として考えていることである。しかし実際には使ってもらえることは前提ではないため、どのようにして使ってもらうのかを考える必要がある。2つ目は、本アプリケーションにユーザにとって魅力的な優位性が必要であることである。認知されていてもユーザにとって使いたいものでなければ使ってもらうことができない。そのため、最終成果発表までにプロモーションの方法を考え、使ってもらうための工夫を本アプリケーションに追加することでユーザを獲得していきたい。
\bunseki{伊藤泰斗}