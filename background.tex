\chapter{背景}

\section{1.1 陣川町について}%例:レビュー内容
必要ならここに大見出しの内容
%必要なら下のsubsectionを用いて小見出しをつかう
%\subsection{ここに小見出し}%:発表技法について
陣川町は北海道函館市にある町である。陣川町には「陣川あさひ町会(以降、町会とする)」がある。町会は陣川町の1,200世帯中約1,000世帯が加入している。また、夏には参加者が役1,000人にもなる納涼まつりや冬にはウィンターフェスティバルを行うなど積極的に活動している。しかし、積極的にイベントを開催する反面でさまざまな問題を抱えている。
\bunseki{船木綾香}

\section{1.2町会が抱える問題}%例:レビュー内容
必要ならここに大見出しの内容
%必要なら下のsubsectionを用いて小見出しをつかう
%\subsection{ここに小見出し}%:発表技法について
町内会のイベントを開催する上での問題点が主に3つ存在する。1つ目の問題点は、イベントの情報が決まった際に同一の内容を2度発信する手間がかかることである。なぜなら、町会はイベント情報を発信するツールとしてFacebookとLINE@の2つを利用しているからである。入力フォーマットが異なるため同時に同一の内容を投稿できない。2つ目の問題点は、イベントの参加者の管理に手間がかかることである。なぜなら、町民のイベントへの参加申し込み方法が電話、FAX、メールの3つあるからである。3つの手段で参加者の情報が町会の元に届くので管理に手間がかかる。3つ目の問題は、イベント当日が悪天候の場合、参加者全員に対してイベントの中止、延期などの連絡を迅速に行うことができないことである。過去のイベントで雨天中止の連絡が出来なかったために、参加者に風邪を引かせてしまったという事例があった。このように、町会はイベントを開催する上で様々な問題を抱えている。
\bunseki{船木綾香}
