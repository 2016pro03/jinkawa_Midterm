\chapter{背景}

\section{1.1 陣川町について}
陣川町は北海道函館市にある町である。陣川町には「陣川あさひ町会(以降、町会とする)」がある。町会は陣川町の1,200世帯中約1,000世帯が加入している。夏には参加者が約1,000人にもなる納涼まつりや冬にはウィンターフェスティバルを行うなど積極的に活動し
ている。また、これらのイベント情報を多くの人に知ってもらいたいため町会役員がFacebookとLINE@を使い発信している。しかし、積極的にイベントを開催する反面で様々な問題を抱えている。
\bunseki{船木綾香}

\section{1.2町会が抱える問題}
町会のイベントを開催する上での問題点は主に以下の6つである。
\begin{itemize}
    \item FacebookやLINE@ではイベントに関するお知らせはできるが、開催予定のイベントを一覧で見れない。
    \item イベントの情報が決まった際にFacebookとLINE@に同一の内容を発信する手間がかかる。
    \item 町民のイベント申し込み方法が電話、FAX、メールの3つあり、イベント参加者の管理に手間がかかる。
    \item Facebookでは個人情報が漏れてしまうため参加申し込みができない。
    \item 役員だけで共有したい情報を町民に知られずに共有することがFacebookやLINE@ではでできない。
    \item イベント当日が悪天候の場合、参加者全員に対してイベントの中止、延期などの連絡を迅速に行うことができない。
\end{itemize}
このように、町会はイベントを開催する上で様々な問題を抱えている。
\bunseki{船木綾香}
