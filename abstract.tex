% 和文概要
\begin{jabstract}
本プロジェクトでは、フィールドを実際に調査してそこから問題点を見つける。そこで見つかった問題点をICTを活用して解決する。それにより地域・社会に貢献することを目標として活動を行っている。開発手法はアジャイル開発手法を用いる。素早くアプリを開発し、それに対するレビューを受けて問題解決の質をより高いものにしていく。
 我々町内会グループは、陣川あさひ町会をフィールドに設定した。函館市陣川町にある陣川あさひ町会(以下、町会とする)は1200世帯中1000世帯が加盟しており1000人規模のイベントを開催しているなど、積極的に活動をしている2016年5月中旬に実際に陣川あさひ町内会へ現地調査へ行き、どのような問題点があるのか、どのような要望があるのかをヒアリングした。ヒアリングした結果、陣川あさひ町会役員(以降、役員とする)が複数のSNSに投稿するのが大変であることや参加者の管理がうまくいっていないこと、イベントの緊急連絡ができていないという問題点があった。そこから我々が話し合って固めたアプリ案をティーチングアシスタント(以降、TAとする)や担当教員、役員の方々からレビューを受けながら、開発を行った。
5月30日の第1回提案では、役員のイベント開催に関する問題を解決するため、カレンダー表示を中心としたイベント管理アプリケーションの提案をした。次に6月23日の第2回提案では第1回提案を経て更に内容を精査して「イベント作成機能」、「イベント参加申し込み機能」、「イベント通知機能」を提案した。しかし7月8日に行われた中間報告会で、町民の方々に使ってもらうための伝達手段について考慮されていない課題が見つかった。そのため8月6日に催される納涼まつりにて町民に対して、どのようにアプリケーションを導入してもらうかを見当して、レビューをいただく予定である。そして使ってもらって学ぶサイクルを繰り返すことで役員の要望にあったシステムデザインを行っていく。

% 和文キーワード
\begin{jkeyword}
陣川町, 陣川あさひ町会, アジャイル開発, アプリケーション, イベント, レビュー, システムデザイン
\end{jkeyword}
\bunseki{伊藤泰斗}
\end{jabstract}
​
%英語の概要
\begin{eabstract} Abstract in English.
In this project, at first, we investigate on the field and find problems from field survey. We solve the problems found from field survey by ICT. Then we have action with the goal of contributing to an area. We use Agile development process which is a software development technique. We do swift app development, and develop higher quality app by being reviewed for it.
% 英文キーワード
\begin{ekeyword}
Keyrods1, Keyword2, Keyword3, Keyword4, Keyword5
\end{ekeyword}
\bunseki{伊藤泰斗}
\end{eabstract}
